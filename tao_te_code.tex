\documentclass[a4paper,10pt]{book}
\usepackage[T1]{fontenc}
\usepackage[utf8]{inputenc}
\usepackage{german}
\usepackage{color}

\newcommand*\myrule{
  \color{blue}
  \hrulefill
  \normalcolor
}
\newcommand*\myheading[1]{
  \color{blue}
  \textbf{#1}
  \normalcolor
}
\renewcommand*{\rmdefault}{cmbr}

\begin{document}

Eine "Ubertragung von Lao-tses Tao-te-king f"ur Programmierer, auf der Grundlage
der "Ubersetzung von Ernst Schwarz.

\chapter{1}

\begin{verse}
Der Code, den man schreiben kann\\
ist nicht der ewige Code.\\
Den Namen, den man vergeben kann\\
ist nicht der ewige Name.

Das Namenlose ist der Anfang und das Ende.\\
Das Namentragende gebiert die zehntausend Befehle.

Der Wunschlose\\
erkennt die tiefsten Zusammenh"ange.\\
Der Begehrliche\\
erreicht das Vordergr"undige.

Beide entspringen derselben Quelle,\\
und treten nur unterschiedlich in Erscheinung.\\
Gemeinsam geh"oren sie der Tiefe.\\
Wo die Tiefe am tiefsten ist,\\
liegt die Pforte zu aller Einsicht.
\end{verse}

\chapter{2}

\begin{verse}
\end{verse}


\end{document}
