\documentclass[a4paper,10pt,openany]{book}
\usepackage[T1]{fontenc}
\usepackage[utf8]{inputenc}
\usepackage{german}
\usepackage{color}

\newcommand*\myrule{
  \color{blue}
  \hrulefill
  \normalcolor
}
\newcommand*\myheading[1]{
  \color{blue}
  \textbf{#1}
  \normalcolor
}
\renewcommand*{\rmdefault}{cmbr}

\title{Tao Te Code}
\author{Eine "Ubertragung von Lao-tses Tao-te-king f"ur Programmierer,\\
auf der Grundlage verschiedener "Ubersetzungen,\\
von Thomas Herchenr"oder.}

\begin{document}
\maketitle


\chapter{}
\begin{verse}
Der Code, den man schreiben kann,\\
ist nicht der ewige Code.\\
Den Namen, den man vergeben kann,\\
ist nicht der ewige Name.

Das Namenlose ist der Anfang und das Ende.\\
Das Namentragende gebiert die zehntausend Zeilen.

Der Wunschlose\\
erkennt die tiefsten Zusammenh"ange.\\
Der Begehrliche\\
erreicht das vordergr"undige Ziel.

Beide entspringen derselben Quelle,\\
und treten nur unterschiedlich in Erscheinung.\\
Gemeinsam geh"oren sie der Tiefe.\\
Wo die Tiefe am tiefsten ist,\\
liegt die Pforte zu aller Einsicht.
\end{verse}

\chapter{}
\begin{verse}
Weil wir etwas sch"on nennen,\\
so entsteht das H"a"sliche.\\
Weil wir etwas gut nennen,\\
so entsteht das Schlechte.

Denn voll und leer geb"aren einander,\\
leicht und schwer bedingen sich,\\
vorher und nachher jagen einander,\\
Code und Bug bringen einander hervor,\\
Suchen und Finden wechseln sich ab.

Darum tut der Weise ohne Taten,\\
er belehrt ohne Worte.\\
So gedeiht der Code ohne Widerstand.\\
So l"asst er ihn wachsen,\\
und besitzt ihn nicht.\\
Tut und verlangt nichts f"ur sich.\\
Nimmt nichts f"ur sich,\\
was er vollbrachte.

Und da er nichts nimmt,\\
kann er nichts verlieren.
\end{verse}

\chapter{}
\begin{verse}
Verehre nicht Titel,\\
und es gibt keinen Streit im Team.

Sch"atze nicht Statussymbole,\\
und es gibt kein Trachten danach.

Prahle nicht mit Begehrenswertem,\\
und die Sinne des Teams werden nicht verwirrt.

So leitet der Weise:\\
Er richtet den Sinn auf das Wesentliche,\\
nicht auf den Vorteil.\\
Er st"arkt die Einsicht,\\
und schw"acht den Eigennutz.

Er stachelt im Team keinen Wetteifer an,\\
und bei den Klugen keine Voreiligkeit.\\
Bei allem, was er tut,\\
verweilt er im Nicht-Tun,\\
und so entsteht Ordnung.

\end{verse}

\chapter{}
\begin{verse}
Die Einsicht ist wie ein unersch"opfliches Gef"a"s,\\
man sch"opft aus ihr,\\
und dennoch leert sie sich nicht.

Sie mildert die Sch"arfen,\\
l"ost die Knoten,\\
schw"acht den blendenden Glanz,\\
befreit vom Staub.

Die Einsicht verbirgt sich,\\
und ist doch immer gegenw"artig.\\
Ich weiss nicht, woher sie kommt,\\
doch vom Anbeginn des Entwurfes\\
war sie.
\end{verse}

% Chap. 5
\chapter{}
\begin{verse}
Code und Compiler kennen keine R"ucksicht,\\
wie Strohpuppen sind f"ur sie alle unsere Versuche.\\
Der Weise kennt keine Bevorzugung,\\
alle Teammitglieder achtet er gleich.

Was im Raum zwischen Code und Coder ist,\\
gleicht es nicht dem Zwischenraum eines Blasebalgs?\\
Leer und doch unversiegbar,\\
bewegt und immer mehr erzeugend.

Weitschweifiger Code verarmt,\\
k"urzer ist besser, weniger ist mehr.
\end{verse}

\chapter{}
\begin{verse}
Unsterblich ist der tiefe Geist der Einfachheit,\\
ihn nenne ich die Quelle.\\
Seine Pforte ist die Wurzel allen Codes.\\
Er durchzieht alles, allgegenw"artig wirkend\\
und wirkt doch m"uhelos.
\end{verse}

\chapter{}
\begin{verse}
Ewig sind Algorithmus und Code,\\
weil sie nicht um ihrer selbst willen wirken.

So stellt der Weise sich selbst zur"uck\\
und ist den anderen voraus.\\
Wahrt nicht seinen Vorteil\\
und er bleibt ihm bewahrt.\\
Denn ohne Eigensucht\\
vollendet er das eigene.
\end{verse}

\chapter{}
\begin{verse}
H"ochste Mitarbeit ist wie das Wasser.\\
Gut tut es allen Dingen und streitet mit keinem.\\
Das Niedrige, das alle verachten, f"ullt es.\\
So gleicht es dem Weg.

F"ur den Code sei die Programmiersprache gut gew"ahlt,\\
das Wesen des Teams gut ergr"undet,\\
die Mitarbeit gutgesinnt,\\
bei Worten die Wahrheit gut erwogen,\\
beim F"uhren der Stil wohl geordnet,\\
bei der Task der T"uchtigste gut ausgesucht,\\
und zum Handeln die Zeit gut ausgew"ahlt.

Nur wer wie das Wasser mit keinem streitet,\\
ist ohne Leid.
\end{verse}

\chapter{}
\begin{verse}
    Besser ist aufh"oren,\\
    als "uberf"ullen.

    Die Klinge, immerfort gesch"arft,\\
    bleibt nicht Klinge.

    Das Software, mit Tand "ueberladen,\\
    verrottet in den Speichern.

    Glanz und Ehre, mit Hochmut gepaart,\\
    ziehen sich selbst ins Verderben.

    Zur"uckziehen nach getaner Arbeit,\\
    das ist der Weg des Weisen.
\end{verse}

\chapter{}
\begin{verse}
Ohne Gesch"aftigsein, ans Eine sich haltend,\\
kann die Seele sich dann noch zerstreuen?

Die Atemkraft sammelnd, geschmeidig werdend,\\
kann man nicht R"uckkehren zur Einfachheit?

Den Blick l"auternd zur Schau des Wesentlichen,\\
kann man nicht frei werden von Verwirrung?

Das Team lieben, das Projekt ordnen,\\
braucht man dazu Gerissenheit?

Kann sich das Tor zum Erfolg "offnen und schlie"sen\\
ohne Dienen?

Klarheit, die alles durchdringt,\\
braucht sie Betriebsamkeit?

Der Weise l"a"st sie wachsen und n"ahrt sie,\\
l"a"st den Code wachsen und besitzt ihn nicht,\\
tut und verlangt nichts f"ur sich,\\
Beh"uter, nicht Beherrscher.\\
Das ist die tiefste Tugend.
\end{verse}

\chapter{}
\begin{verse}
    Drei"sig Speichen umringen die Nabe.\\
    Wo nichts ist,\\
    liegt der Nutzen des Rades.

    Aus Ton formt ein T"opfer den Topf.\\
    Wo er hohl ist,\\
    liegt der Nutzen des Topfes.

    T"ur und Fenster h"ohlen die W"ande.\\
    Wo es leer bleibt,\\
    liegt der Nutzen des Hauses.

    So bringt Seiendes Gewinn,\\
    aber Nichtseinendes Nutzen.
\end{verse}

\chapter{}
\begin{verse}
    Pracht blendet das Auge,\\
    Klangf"ulle bet"aubt das Ohr.\\
    Feinschmeckerei verdirbt den Geschmack,\\
    Hetzen und Jagen verwirren das Herz.\\
    Seltene G"uter f"uhren zu Begehrlichkeiten.

    Darum sorgt der Weise f"uhr das Wesentliche,\\
    nicht f"ur das "Uberfl"ussige,\\
    f"ur jenes, nicht f"ur dieses.
\end{verse}

\chapter{}
\begin{verse}
    Gunst und Benachteiligung\\
    - Angst machen beide.\\
    Ein grosses "Ubel wie das Ego wird hochgesch"atzt.

    Warum sage ich: Gunst und Benachteiligung\\
    - Angst machen beide?\\
    Gunst gilt dem Tieferstehenden,\\
    "angstlich empf"angt er sie,\\
    mit Angst verliert er sie.\\
    So sage ich: Gunst und Benachteiligung\\
    - Angst machen beide.

    Und warum sage ich: Ein grosses "Ubel wie das Ego\\
    wird hochgesch"atzt?\\
    Befallen werde ich von gro"sen "Ubeln,\\
    weil ich ein Ego habe.\\
    W"are ich frei vom Ego,\\
    welches "Ubel g"abe es f"ur mich?

    Dem aber, der die Welt zu seinem Eigentlichen macht,\\
    mag man das Projekt "ueberlassen.\\
    Dem, der das Ganze liebend dem eigenen Selbst gleichsetzt,\\
    mag man die Aufgabe anvertrauen.
\end{verse}

% ------------------------------------------------------------------------------
% APPENDIX
% ------------------------------------------------------------------------------

\appendix
\chapter{"Ubersetzungen}

Die folgenden "Ubersetzungen wurden benutzt:

\begin{description}
\item Schwarz, Ernst, \textit{Laudse Daudedsching}, Deutscher Taschenbuch Verlag, M"unchen 1988\textsuperscript{3}
\item Debon, G"unther, \textit{Lao-tse Tao-Te-King}, Phillip Reclam jun., Stuttgart 2012
\item Knospe, Hans; Br"andli, Odette, \textit{Lao Tse Tao-Te-King}, Diogenes, Z"urich 1996
\end{description}

\end{document}
