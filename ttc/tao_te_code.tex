\documentclass[a4paper,10pt,openany]{book}
\usepackage[T1]{fontenc}
\usepackage[utf8]{inputenc}
\usepackage{german}
\usepackage{color}

\newcommand*\myrule{
  \color{blue}
  \hrulefill
  \normalcolor
}
\newcommand*\myheading[1]{
  \color{blue}
  \textbf{#1}
  \normalcolor
}
\renewcommand*{\rmdefault}{cmbr}

\title{Tao Te Code}
\author{Eine "Ubertragung von Lao-tses Tao-te-king f"ur Programmierer,\\
auf der Grundlage verschiedener "Ubersetzungen,\\
von Thomas Herchenr"oder.}

\begin{document}
\maketitle


\chapter{}
\begin{verse}
Der Code, den man schreiben kann,\\
ist nicht der ewige Code.\\
Den Namen, den man vergeben kann,\\
ist nicht der ewige Name.

Das Namenlose ist der Anfang und das Ende.\\
Das Namentragende gebiert die zehntausend Zeilen.

Der Wunschlose\\
erkennt die tiefsten Zusammenh"ange.\\
Der Begehrliche\\
erreicht das vordergr"undige Ziel.

Beide entspringen derselben Quelle,\\
und treten nur unterschiedlich in Erscheinung.\\
Gemeinsam geh"oren sie der Tiefe.\\
Wo die Tiefe am tiefsten ist,\\
liegt die Pforte zu aller Einsicht.
\end{verse}

\chapter{}
\begin{verse}
Weil wir etwas sch"on nennen,\\
so entsteht das H"a"sliche.\\
Weil wir etwas gut nennen,\\
so entsteht das Schlechte.

Denn voll und leer geb"aren einander,\\
leicht und schwer bedingen sich,\\
vorher und nachher jagen einander,\\
Code und Bug bringen einander hervor,\\
Suchen und Finden wechseln sich ab.

Darum tut der Weise ohne Taten,\\
er belehrt ohne Worte.\\
So gedeiht der Code ohne Widerstand.\\
So l"asst er ihn wachsen,\\
und besitzt ihn nicht.\\
Tut und verlangt nichts f"ur sich.\\
Nimmt nichts f"ur sich,\\
was er vollbrachte.

Und da er nichts nimmt,\\
kann er nichts verlieren.
\end{verse}

\chapter{}
\begin{verse}
Verehre nicht Titel,\\
und es gibt keinen Streit im Team.

Sch"atze nicht Statussymbole,\\
und es gibt kein Trachten danach.

Prahle nicht mit Begehrenswertem,\\
und die Sinne des Teams werden nicht verwirrt.

So leitet der Weise:\\
Er richtet den Sinn auf das Wesentliche,\\
nicht auf den Vorteil.\\
Er st"arkt die Einsicht,\\
und schw"acht den Eigennutz.

Er stachelt im Team keinen Wetteifer an,\\
und bei den Klugen keine Voreiligkeit.\\
Bei allem, was er tut,\\
verweilt er im Nicht-Tun,\\
und so entsteht Ordnung.

\end{verse}

\chapter{}
\begin{verse}
Die Einsicht ist wie ein unersch"opfliches Gef"a"s,\\
man sch"opft aus ihr,\\
und dennoch leert sie sich nicht.

Sie mildert die Sch"arfen,\\
l"ost die Knoten,\\
schw"acht den blendenden Glanz,\\
befreit vom Staub.

Die Einsicht verbirgt sich,\\
und ist doch immer gegenw"artig.\\
Ich weiss nicht, woher sie kommt,\\
doch vom Anbeginn des Entwurfes\\
war sie.
\end{verse}

% Chap. 5
\chapter{}
\begin{verse}
Code und Compiler kennen keine R"ucksicht,\\
wie Strohpuppen sind f"ur sie alle unsere Versuche.\\
Der Weise kennt keine Bevorzugung,\\
alle Teammitglieder achtet er gleich.

Was im Raum zwischen Code und Coder ist,\\
gleicht es nicht dem Zwischenraum eines Blasebalgs?\\
Leer und doch unversiegbar,\\
bewegt und immer mehr erzeugend.

Weitschweifiger Code verarmt,\\
k"urzer ist besser, weniger ist mehr.
\end{verse}

\chapter{}
\begin{verse}
Unsterblich ist der tiefe Geist der Einfachheit,\\
ihn nenne ich die Quelle.\\
Seine Pforte ist die Wurzel allen Codes.\\
Er durchzieht alles, allgegenw"artig wirkend\\
und wirkt doch m"uhelos.
\end{verse}

\chapter{}
\begin{verse}
Ewig sind Algorithmus und Code,\\
weil sie nicht um ihrer selbst willen wirken.

So stellt der Weise sich selbst zur"uck\\
und ist den anderen voraus.\\
Wahrt nicht seinen Vorteil\\
und er bleibt ihm bewahrt.\\
Denn ohne Eigensucht\\
vollendet er das eigene.
\end{verse}

\chapter{}
\begin{verse}
H"ochste Mitarbeit ist wie das Wasser.\\
Gut tut es allen Dingen und streitet mit keinem.\\
Das Niedrige, das alle verachten, f"ullt es.\\
So gleicht es dem Weg.

F"ur den Code sei die Programmiersprache gut gew"ahlt,\\
das Wesen des Teams gut ergr"undet,\\
die Mitarbeit gutgesinnt,\\
bei Worten die Wahrheit gut erwogen,\\
beim F"uhren der Stil wohl geordnet,\\
bei der Task der T"uchtigste gut ausgesucht,\\
und zum Handeln die Zeit gut ausgew"ahlt.

Nur wer wie das Wasser mit keinem streitet,\\
ist ohne Leid.
\end{verse}

\chapter{}
\begin{verse}
    Besser ist aufh"oren,\\
    als "uberf"ullen.

    Die Klinge, immerfort gesch"arft,\\
    bleibt nicht Klinge.

    Das Software, mit Tand "ueberladen,\\
    verrottet in den Speichern.

    Glanz und Ehre, mit Hochmut gepaart,\\
    ziehen sich selbst ins Verderben.

    Zur"uckziehen nach getaner Arbeit,\\
    das ist der Weg des Weisen.
\end{verse}

% Chap. 10
\chapter{}
\begin{verse}
Ohne Gesch"aftigsein, ans Eine sich haltend,\\
kann die Seele sich dann noch zerstreuen?

Die Atemkraft sammelnd, geschmeidig werdend,\\
kann man nicht R"uckkehren zur Einfachheit?

Den Blick l"auternd zur Schau des Wesentlichen,\\
kann man nicht frei werden von Verwirrung?

Das Team lieben, das Projekt ordnen,\\
braucht man dazu Gerissenheit?

Kann sich das Tor zum Erfolg "offnen und schlie"sen\\
ohne Dienen?

Klarheit, die alles durchdringt,\\
braucht sie Betriebsamkeit?

Der Weise l"a"st sie wachsen und n"ahrt sie,\\
l"a"st den Code wachsen und besitzt ihn nicht,\\
tut und verlangt nichts f"ur sich,\\
Beh"uter, nicht Beherrscher.\\
Das ist die tiefste Tugend.
\end{verse}

\chapter{}
\begin{verse}
    Drei"sig Speichen umringen die Nabe.\\
    Wo nichts ist,\\
    liegt der Nutzen des Rades.

    Aus Ton formt ein T"opfer den Topf.\\
    Wo er hohl ist,\\
    liegt der Nutzen des Topfes.

    T"ur und Fenster h"ohlen die W"ande.\\
    Wo es leer bleibt,\\
    liegt der Nutzen des Hauses.

    So bringt Seiendes Gewinn,\\
    aber Nichtseinendes Nutzen.
\end{verse}

\chapter{}
\begin{verse}
    Pracht blendet das Auge,\\
    Klangf"ulle bet"aubt das Ohr.\\
    Feinschmeckerei verdirbt den Geschmack,\\
    Hetzen und Jagen verwirren das Herz.\\
    Seltene G"uter f"uhren zu Begehrlichkeiten.

    Darum sorgt der Weise f"uhr das Wesentliche,\\
    nicht f"ur das "Uberfl"ussige,\\
    f"ur jenes, nicht f"ur dieses.
\end{verse}

\chapter{}
\begin{verse}
    Gunst und Benachteiligung\\
    - Angst machen beide.\\
    Ein grosses "Ubel wie das Ego wird hochgesch"atzt.

    Warum sage ich: Gunst und Benachteiligung\\
    - Angst machen beide?\\
    Gunst gilt dem Tieferstehenden,\\
    "angstlich empf"angt er sie,\\
    mit Angst verliert er sie.\\
    So sage ich: Gunst und Benachteiligung\\
    - Angst machen beide.

    Und warum sage ich: Ein grosses "Ubel wie das Ego\\
    wird hochgesch"atzt?\\
    Befallen werde ich von gro"sen "Ubeln,\\
    weil ich ein Ego habe.\\
    W"are ich frei vom Ego,\\
    welches "Ubel g"abe es f"ur mich?

    Dem aber, der die Welt zu seinem Eigentlichen macht,\\
    mag man das Projekt "ueberlassen.\\
    Dem, der das Ganze liebend dem eigenen Selbst gleichsetzt,\\
    mag man die Aufgabe anvertrauen.
\end{verse}

\chapter{}
\begin{verse}
    Das Auge sieht es nicht - ihr nennt es unsichtbar.\\
    Das Ohr h"ort es nicht - ihr nennt es unh"orbar.\\
    Die Hand fa"st es nicht - ihr nennt es unfa"sbar.\\
    Dreifach trotzt es dem Verstand,\\
    denn es ist eines, in sich selbst verwoben.

    Oben ohne Licht, unten ohne Dunkelheit.\\
    Es dehnt sich hin unendlich, namenlos,\\
    und str"omt zur"uck in das Nichtdingliche.\\
    So nenne ich es gestaltlose Gestalt,\\
    Ding der Nichtdinglichkeit.

    Nennen mag man es formlos, nebelhaft.\\
    Entgegentretend sieht man nicht sein Gesicht,\\
    ihm folgend nicht seinen R"ucken.

    Haltet fest am Dao der Alten,\\
    leitet mit ihm das Neue.\\
    Den Uranfang zu erkennen,\\
    nenne ich Leitspur des Dao.
\end{verse}

% Chap. 15
\chapter{}
\begin{verse}
    Die wahrhaft Verst"andigen des Projektes,\\
    feingeistig sind sie, Verborgenes durchdringend,\\
    tief, unbegreifbar tief.

    Und da sie nicht zu begreifen sind,\\
    mu"s man sie in Bildern beschreiben:

    Z"ogernd und vorsichtig,\\
    als gingen sie "uber einen gefrorenen Fluss.\\
    Wachsam, als m"ussten sie sich vor den Nachbarn h"uten.\\
    Zur"uckhaltend, wie ein Gast.\\
    Nachgiebig, wie schmelzendes Eis.\\
    Schlicht und einfach, wie ein unbehauener Holzklotz.\\
    Weit und leer, wie ein Tal.\\
    Undurchsichtig, wie schlammiges Gew"asser.

    Wer kann allm"ahlich Tr"ubes kl"aren durch Ruhe?\\
    Wer kann Langbewegtes beruhigen und allm"ahlich\\
    zu Wachstum bringen?

    Wer das Dao bewahrt, begehrt nicht "Uberf"ulle.\\
    Wer nicht "Uberf"ulle begehrt,\\
    kann erhalten, ohne Neues zu schaffen.
\end{verse}

\chapter{}
\begin{verse}
    Erreiche den Gipfel der Leere,\\
    bewahre die F"ulle der Ruhe,\\
    und alle Dinge werden gedeihen.

    So kann ich ihre R"uckkehr erschauen.\\
    Von allen Dingen in ihrer Vielfalt\\
    findet ein jedes zur"uck zur Wurzel.\\
    Wurzelwiederfinden hei"st Stille -\\
    was man nennen mag:\\
    R"uckkehr zum Wesen.

    R"uckkehr zum Wesen hei"st Ewigdauern.\\
    Ewigdauerndes kennen hei"st Klarheit.\\
    Wer Ewigdauerndes nicht kennt,\\
    wirkt blindlings zum Unheil.

    Wer Ewigdauerndes kennt, umfa"st alles.\\
    Wer alles umfa"st, geh"ort allen.\\
    Wer allen geh"ort, ist vorbildlich.

    Vorbildliches gleicht dem Himmel.\\
    Der Himmel gleicht dem Dao,\\
    das Dao gleicht der Ewigkeit.\\
    Wer im Dao verweilt,\\
    taucht gefahrlos in die Tiefe.
\end{verse}

\chapter{}
\begin{verse}
    Zuerst wu"sten die Coder kaum von Managern.\\
    Sp"ater dr"angten sie sich um sie und r"uhmten sie.\\
    Sie zu f"urchten lernten sie sp"ater,\\
    dann zu verachten.

    Wo das Vertrauen fehlt,\\
    spricht der Verdacht.\\
    Wahre Manager legen nicht Wert auf Worte,\\
    von Wert sind alleine ihre Taten.\\
    Von selbst getan erscheinen sie dem Team.
\end{verse}

\chapter{}
\begin{verse}
    Die gro"se Einsicht ging verloren,\\
    und es entstanden Eifer und Rechtschaffenheit.

    Die Klugheit trat hervor,\\
    und es entstand Heuchelei.

    Das Team wurde zerrissen,\\
    und Teamgeist entstand.

    Das Projekt zerfiel in Wirrnissen,\\
    und die Methodik entstand.
\end{verse}

\chapter{}
\begin{verse}
    Schafft den Perfektionismus ab,\\
    verwerft die Cleverness,\\
    und das Team wird hundertfach gewinnen.

    Schafft den Eifer ab,\\
    verwerft die Rechtschaffenheit,\\
    und die Menschen werden wieder aufrichtig\\
    miteinander arbeiten.

    Schafft die oberfl"achliche Geschicklichkeit ab,\\
    verwerft die Gewinnsucht,\\
    und es wird keine Einbr"uche im Projekt mehr geben.

    Denn das alles ist doch nur oberfl"achlich und taugt nichts.

    Darum lehre die Mitarbeiter:\\
    Zum Schlichten und Simplen zur"uckkehren,\\
    das Einfache anstreben\\
    statt mit Features "uberfrachten.
\end{verse}

% Chap. 20
\chapter{}
\begin{verse}
    La"s das Bekannte immer wieder hinter dir,\\
    und du wirst ohne Sorgen sein.

    Wie wenig trennt Jasagen von Heuchelei,\\
    wie wenig scheidet gut von schlecht.

    Soll ich das f"urchten,\\
    was alle anderen f"urchten?\\
    Endlos scheinen die Wirrsalen des Projekts.

    Die Mitarbeiter benehmen sich,\\
    als gingen sie zum Betriebsfest,\\
    als z"ogen sie schon zur Releaseparty.

    Der Weise allein ruht in sich, still und ungeteilt.\\
    Als ob er ein Kind ist,\\
    um das sich noch die Mutter k"ummert.\\
    Ein Ungebundener, der sich an nichts klammert.

    Alle haben mehr, als sie brauchen.\\
    Der Weise gibt leichten Herzens.\\
    Sein Herz gleicht dem Herz eines Toren,\\
    schlicht und unbek"ummert.

    Die Menge liebt den Glanz,\\
    der Weise allein liebt das Unsp"aktakul"are.\\
    Die Menge liebt die Unterscheidung,\\
    der Weise allein liebt das Unterschiedslose.\\
    Bewegt wie das Meer,\\
    ziellos wie der Wind.

    Alle tun als w"aren sie von Nutzen,\\
    nur der Weise k"ummert sich nicht um sein Image.\\
    Er ist anders als die anderen,\\
    und h"alt sich ans Eigentliche.
\end{verse}

\chapter{}
\begin{verse}
    Die gr"o"ste Tugend ist es,\\
    dem Weg zu folgen.

    Das, was Weg genannt wird,\\
    ist unfa"sbar und unvorstellbar.\\
    Und doch ist in ihm ein Bild.\\
    Und doch ist in ihm ein Wesen.\\
    Unergr"undlich und dunkel,\\
    und doch ist in ihm ein Geist.

    Sein Geist ist die Wirklichkeit,\\
    darauf kann man sich st"utzen.\\
    Er kann nicht vergessen werden,\\
    weil der Anfang aller Dinge darin liegt.

    Wie erkenne ich, da"s ihm alle Dinge entspringen?\\
    Durch eben ihn.
\end{verse}

\chapter{}
\begin{verse}
    Verkr"uppeltes wird ganz,\\
    Krummes wird gerade.\\
    Leeres wird voll,\\
    Altes wird neu.\\
    Wenig wird viel,\\
    Vieles macht wirr.

    So h"alt sich der Weise ans Eine,\\
    und wird zum Vorbild f"ur alle.\\
    Er zeigt sich nicht,\\
    so wird er sichtbar.\\
    Er will nicht recht behalten,\\
    so wird sein Recht offenbar.\\
    Er pocht nicht auf Verdienste,\\
    so schafft er Verdienstvolles.\\
    Er tut sich nicht hervor,\\
    so f"allt ihm der Vorrang von selber zu.

    Nur wer mit keinem streitet,\\
    bleibt unbestritten Sieger.

    So ist das Wort der Alten,\\
    ''Verkr"uppeltes wird ganz'',\\
    kein leeres Gerede.\\
    Was wahrhaft ganz wird,\\
    dem str"omt alles zu.
\end{verse}


% ------------------------------------------------------------------------------
% APPENDIX
% ------------------------------------------------------------------------------

\appendix
\chapter{"Ubersetzungen}

Die folgenden "Ubersetzungen wurden benutzt:

\begin{description}
\item Schwarz, Ernst, \textit{Laudse Daudedsching}, Deutscher Taschenbuch Verlag, M"unchen 1988\textsuperscript{3}
\item Debon, G"unther, \textit{Lao-tse Tao-Te-King}, Phillip Reclam jun., Stuttgart 2012
\item Knospe, Hans; Br"andli, Odette, \textit{Lao Tse Tao-Te-King}, Diogenes, Z"urich 1996
\end{description}

\end{document}
